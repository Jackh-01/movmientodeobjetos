\documentclass{article}
\usepackage[utf8]{inputenc}

\title{Como llevar un objeto de la posición A a la B}
\author{jackh.narvaez }
\date{March 2021}

\usepackage{natbib}
\usepackage{graphicx}

\begin{document}

\maketitle

\section{Introduccion}
con el fin de construir una especie de triangulo encima de una hoja hemos diseñado estas instrucciones para que aquella persona que este interesada en hacerlo lo practique, cabe aclarar que este reto sera llevado a cabo con una sola mano.

\section{Instrucciones}
pasos a seguir:
\begin{enumerate}
    \item tome la hoja y pongala en un espacio libre.  
    \item tome las dos tarjetas de tal manera que quede una sobre la otra. 
    \item cuando tenga las tarjetas en el aire tomelas por el lado mas largo.
    \item  sostenlas con tu dedo medio y pulgar.
    \item  llevalas hasta que este encima de la hoja y deja que reposen  las tarjetas sobre la hoja. 
    \item  con el dedo indice vaya abriendolas hasta el punto de que se sostengan por si solas.
\end{enumerate}

\section{Conclusion}
Este reto nos enseño que unas instrucciones puede ser visto de diferentes maneras, si las instrucciones estan bien redactadas puede suceder que el objetivo sea el deseado. 

\end{document}
